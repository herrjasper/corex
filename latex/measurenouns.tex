This case study represents a replication of \citet{Schaefer2018}.
The authors models case alternation in German measure noun phrases (MNPs) based on data from the DECOW corpus \citep{SchaeferBildhauer2012}.
Only MNPs with a kind-denoting noun (such as \textit{Wein} `wine') and a measure noun (such as \textit{Glas} `glass') similar to English \textit{a glass of wine} are considered.
In the first alternant, the kind noun is in the genitive as in (\ref{ex:intro:alternation1}).
In the second alternant, the kind noun shows case identity with the head measure noun as in (\ref{ex:intro:alternation2}).%
\footnote{Examples are taken from \citet[735]{Schaefer2018}.}

\begin{exe}
  \ex\label{ex:intro:alternation}
  \begin{xlist}
    \ex \gll Wir trinken [[ein Glas]\Sub{Acc} [guten Weins]\Sub{Gen}]\Sub{Acc}.\\
    we drink a glass good wine \\
    \trans We drink a glass of good wine.\label{ex:intro:alternation1}
    \ex Wir trinken [[ein Glas]\Sub{Acc} [guten Wein]\Sub{Acc}].\label{ex:intro:alternation2}
  \end{xlist}
\end{exe}

It is noteworthy that the alternation only occurs if the embedded kind noun is determinerless but is modified by an adjective \citep[737--738]{Schaefer2018}.
Based on theoretical analysis, the author names a number of influencing factors for this alternation such as whether the whole MNP has a cardinal determiner, which case it stands in as a whole, and how strongly the individual lexical items favour a specific variant.
Based on previously published assumptions (\eg, \citealt{Hentschel1993,Zimmer2015}), he also models the effect of register or style.
It is generally assumed that certain registers or styles favour the genitive alternant over the alternant with case identity.
However, given the annotation available in the corpus, only two weak indicator variables are used as proxies, namely \texttt{Badness} (a measure of document quality available for all DECOW documents; see \citealt{SchaeferEa2013}) and \texttt{Genitives} (a measure of the frequency of genitives).
Only \textit{Genitives} passes the tests in the original study, and it is a weak effect \citep[749--751]{Schaefer2018}.
These indicator variables are effectively like a highly reduced COReX feature set, and we attempt replicate the study here in our exploration of the usefulness of the COReX features.
