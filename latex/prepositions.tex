

A number of prepositions in German show variation in assigning  case to their NP complement.
A well-known example is the accusative/dative alternation after certain prepositions, which systematically encodes a semantic distinction (directional/non-directional movement; REF).
Contrasting with this kind of semantically relevant alternation, some prepositions exhibit variation in case assignment which is not semantically motivated, but rather considered stylistic.
The present case study focuses on alternations of the second type, involving genitive and dative case, as illustrated in (\ref{prep-example1})--(\ref{prep-example4}).

\begin{exe}
  \ex \label{prep-example1}
    \begin{xlist}
      \ex trotz [starkem Verkehr]$_{dat}$\\
          `despite heavy traffic'
      \ex trotz [ihres Namens]$_{gen}$\\ 
          `despite her name'
    \end{xlist}
  \ex
  \begin{xlist}
    \ex wegen [dem Geschmack]$_{dat}$\\
        `because of the taste'
    \ex wegen [des besseren Aussehens]$_{gen}$\\ 
        `because of the better appearance' 
  \end{xlist}
  \ex
    \begin{xlist}
      \ex entgegen [dem urspr\"unglichen Gesetzentwurf]$_{dat}$\\
          `contrary to the original bill' 
      \ex entgegen [des Gesamttrends]$_{gen}$\\ 
          `contrary to the overall trend'
  \end{xlist}
  \ex \label{prep-example4}
    \begin{xlist}
        \ex gegen\"uber [einem Dritten]$_{dat}$\\
            `vis-à-vis a third party'
        \ex gegen\"uber [des Hotels]$_{gen}$\\ 
            `opposite the hotel' 
    \end{xlist}  
\end{exe}


 Typically, one of the variants is considered as normative / canonical in Standard German, while the competing form has in many cases a non-standard flavour (see e.\,g., \citealp{Dimeola2009} and references therein).\footnotemark\ 
  We therefore expect that the choice of case after these prepositions depends partially on register and, more specifically, on the dimension of variation captured by Factor~1.
Such case alternations thus provide a promising area for exploring the effects of feature aggregation in modeling register-sensitive linguistic alternation phenomena.


