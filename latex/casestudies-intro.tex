\section{Case studies}
\label{sec:case-studies}
In this section, we explore the consequences of aggregating individual linguistic predictors into more abstract factors, for the purpose of modeling linguistic alternation phenomena.
In a \marginpar{series?}series of case studies, we model particular morpho-syntactic alternation phenomena with generalized linear models (GLMs), using different sorts of document-level information as predictors. For each case, we specify two alternative models, estimate the model coefficients, and compare these models wrt.\ to model fit and prediction accuracy:

\begin{enumerate}
  \item a model which uses as predictors the 7 factor scores from the factor analysis described in the previous section
  \item a model which uses as predictors the set of individual COReX features
\end{enumerate}

We are aware that these linguistic phenomena could probably be better explained / modeled if other kinds of predictors were taken into account as well (e.\,g., lexical information, syntactic properties at various levels).
However, since we are interested in what different sorts of document-level information can contribute to modeling the alternation phenomena, we deliberately ignore other predictors as part of our study design.

