\section{Case studies}
\label{sec:case-studies}
In this section, we explore the consequences of aggregating individual linguistic predictors into more abstract factors, for the purpose of modeling linguistic alternation phenomena. In a series of case studies, we model particular morpho-syntactic alternation phenomena with generalized linear models (GLMs), using only document-level information. Each time, we use as predictors the full set of linguistic feature counts as extracted by COReX. In addition, we specify an alternative model using as predictors the per-document factor loadings from a factor analysis. Finally, we compare these models wrt. to model fit and prediction accuracy. 
We are aware that some or all of the phenomena could probably be better explained / modelled if other kinds of predictors were taken into account as well (e.g., lexical information, syntactic properties at various levels). However, since we are interested in what different sorts of document-level information can contribute to modelling the alternation phenomena, we deliberately ignore other predictors as part of our study design.

