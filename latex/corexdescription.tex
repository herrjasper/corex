\section{The COReX feature extractor}

COReX is a piece of software that extracts a large number of normalised linguistic feature counts at the document level from German text.1 It does not perform any linguistic annotation by itself, but instead requires linguistic pre-processing, typically performed automatically by dedicated annotation tools. The input format is vertical text (one token per line) with token level annotations in columns 2 through n, and token spans represented as XML-elements (this corresponds to the input format required by tools such as OpenCWB and NoSketchEngine). Required XML-elements are doc (document) and s (sentence). In order to extract the full range of features, the data must include part-of-speech tags (STTS, ), morphological features (such as those produced by MarMoT,\citealp{MuellerSchmidSchuetze2013}), named entity annotations (such as those produced by the Stanford NER tagger, \citealp{FinkelGrenagerManning2005}) as well as topological field annotations (Berkeley, REF). For some annotation layers, such as morphological features and topological fields, a particular formatting is required. COReX outputs a modified version of the input data where normalised document-level feature counts are added as attribute-value pairs to the opening <doc > tag for each document. Moreover, a number of non-normalised counts (such as perfect and passive) are added as attribute-value pairs to the <s > tag for each sentence. Features include non-linguistic categories such as mean word length and mean sentence length; counts for individual parts-of-speech; ...
 
[Distribution of features in COW data; clustering]
