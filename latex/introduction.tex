\section{Introduction}

Motivation: Variation, Biber, Bigbert \& automatic register classification based on document-level counts of linguistic features; efforts to automatically generate document-level genre/register meta data based on such data.
Aim: demonstrate more appropriate use of linguistic features in variational linguistic research;
Outline: Present Biber-like feature extraction software for German; extract document-level features from corpora of newspaper and web texts; aggregate features via factor analysis; 
In 3 short case studies of (possibly register sensitive) morpho-syntactic variation phenomena, compare the ``explanatory'' power of the resulting factors to that of (selected) raw features, i.e., not using aggregation. 

Clarifications:
\begin{enumerate}
  \item The technical and statistical points made in this paper are not new; our aim is rather to demonstrate the impact of particular aggregation techniques in the context of variationist corpus linguist research, and discuss some of its consequences for corpus construction and corpus annotation. 

  \item We do \textit{not} advocate ``fishing'', ``snooping'', ``hunting'': in practice, if the  study is not explicitly declared as explorative, researchers should approach a given phenomenon with clear idea of which variables are relevant, based on ``substantive theory''.

  \item The case studies presented here are merely intended to highlight the differences between models using aggregated data and models using non-aggregated data. So we focus on the automatically extracted linguistic features, deliberately neglecting other factors that can be expected to play a role in modeling a given phenomenon.
\end{enumerate}
