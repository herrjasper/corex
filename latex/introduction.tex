\section{Alternation modelling and register classification}

\subsection{Corpus-based modelling of linguistic variation}

Modelling linguistic variation in the broadest sense is clearly one of the major goals -- if not \textit{the} major goal -- of linguistics.
Speakers trivially do not use the same expressions for non-identical content (compositional semantics) or communicative intent (pragmatics).
They do not use the same expressions even for the same content or communicative intent in different regions (regio- or dialect), different social groups (sociolect), or based on individual preference (style).
Furthermore, speakers use different forms in different communicative settings (such as talking to close personal friends vs.\ giving a talk in the process of applying for a tenured academic position), which is what we understand as \textit{register}.
Linguistic variation -- even under a fairly traditional view -- thus occurs at the level of individual utterances (driven by semantics and pragmatics), the level of individual speakers (style), the level of groups of speakers (dialect and sociolect), and the within-speaker level (style).
Furthermore, usage-based approaches, which have been on a powerful rise for fifteen to twenty years now, stress that the distribution of concrete forms in utterances is also driven partially by input co-occurrence frequencies.
Under this view, variation is also biased by the fact that some forms happen to be more frequently used together than others, be it for obvious functional or less transparent or obscured reasons.
Speakers most likely also chose their forms partially based on such co-occurrence preferences.%
\footnote{It is clear at this point that all dimensions of variation might overlap and be difficult to disentangle.
Also, variation based on co-occurrence frequencies does not represent an independent dimension of variation as long as there is a transparent functional motivation for a co-occurrence preference.
However, the usage-based perspective adds a stochastic and input-based tone to the picture.}
Finally, random variation should also be expected, either as an effect of so-called \textit{performance} (\ie, production errors based on processing limitations) or because the linguistic system itself (at least as represented in the brains of speakers) does not represent a discrete (traditional linguistic competence) but rather a stochastic system (probabilistic grammar).%
\footnote{Luckily, the heated debate between those two views of random variation in linguistic output is not relevant to this study.}
Modelling linguistic variation thus subsumes the task of specifying why speaker uses the specific expressions that they use under any given circumstances.
Thus, if all sources of variation had been pinpointed and modelled, a full model of human language  would emerge under this broad view.


\subsection{Potentials of incorporating register}



\subsection{The problems with registers in corpora}


\subsection{Statistical issues}



\subsection{Goals and overview}

\cite[vgl.][33--36]{Mueller2010}


% Motivation: Variation, Biber, Bigbert \& automatic register classification based on document-level counts of linguistic features; efforts to automatically generate document-level genre/register meta data based on such data.
% Aim: demonstrate more appropriate use of linguistic features in variational linguistic research;
% Outline: Present Biber-like feature extraction software for German; extract document-level features from corpora of newspaper and web texts; aggregate features via factor analysis; 
% In 3 short case studies of (possibly register sensitive) morpho-syntactic variation phenomena, compare the ``explanatory'' power of the resulting factors to that of (selected) raw features, i.e., not using aggregation. 

\newpage
